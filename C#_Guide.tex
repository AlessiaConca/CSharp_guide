\documentclass[a4paper,12pt]{book}
\usepackage{listings}
\usepackage{hyperref}
\usepackage{amsmath}
\usepackage{graphicx}
\usepackage{xcolor}

\title{C\# Developer for Game Programming}
\author{Alessia C.R.}
\date{\today}

% Custom settings for C# language highlighting
\lstdefinelanguage{CSharp}{
	morekeywords={abstract, as, base, bool, break, byte, case, catch, char, checked, class, const, continue, decimal, default, delegate, do, double, else, enum, event, explicit, extern, false, finally, fixed, float, for, foreach, goto, if, implicit, in, int, interface, internal, is, lock, long, namespace, new, null, object, operator, out, override, params, private, protected, public, readonly, ref, return, sbyte, sealed, short, sizeof, stackalloc, static, string, struct, switch, this, throw, true, try, typeof, uint, ulong, unchecked, unsafe, ushort, using, virtual, void, volatile, while},
	sensitive=true,
	morecomment=[l]{//},
	morecomment=[s]{/*}{*/},
	morestring=[b]",
}

\lstset{
	language=CSharp,
	basicstyle=\ttfamily\small,
	keywordstyle=\bfseries\color{blue},
	commentstyle=\itshape\color{gray},
	stringstyle=\color{red},
	numbers=left,
	numberstyle=\tiny\color{gray},
	stepnumber=1,
	numbersep=10pt,
	backgroundcolor=\color{white},
	tabsize=4,
	showspaces=false,
	showstringspaces=false,
	breaklines=true
}

\begin{document}
	
	\maketitle
	\tableofcontents
	
	\chapter{Introduction to C\# for Game Development}
	\section{What is C\#?}
	C\# is a modern, object-oriented programming language developed by Microsoft. It is designed to be a simple, general-purpose, object-oriented language that enables developers to build a wide range of applications, including games. C\# is commonly used in game development for its powerful features, performance optimizations, and its close integration with game development environments like Unity and MonoGame.
	
	\section{Why Use C\# for Game Development?}
	\subsection{Strong Ecosystem and Unity Integration}
	One of the main reasons C\# is popular for game development is its seamless integration with Unity, one of the most widely used game engines in the industry. Unity is free to use, cross-platform, and supports both 2D and 3D game development, making it an attractive option for developers. MonoGame, another framework, is also commonly used for building 2D and 3D games in C\#.
	
	\subsection{Ease of Learning and Rich Documentation}
	C\# is an easy language to learn, especially for those with experience in other object-oriented languages like Java or C++. The language comes with robust documentation, tutorials, and community support, which helps beginners get started with building games quickly.
	
	\section{Setting Up Your Development Environment}
	\subsection{Installing Visual Studio}
	Visual Studio is the recommended integrated development environment (IDE) for C\# development. Visual Studio comes with built-in tools for writing, testing, and debugging C\# code. You can download the free Community edition from the official Visual Studio website.
	
	\subsection{Installing Unity and MonoGame}
	\textbf{Unity}: Unity is a powerful, cross-platform game engine that supports both 2D and 3D development. It is the most popular game development platform for C\# developers. Unity provides an editor for managing assets, game objects, and scripts.
	
	\textbf{MonoGame}: MonoGame is an open-source game development framework that allows developers to build cross-platform games using C\#. It is suitable for both 2D and 3D games and is a great alternative for those not using Unity.
	
	\section{First Steps in Unity}
	Once Unity is installed, open it and create a new project. The Unity Editor is where you'll design game levels, control object interactions, and write C\# scripts. In Unity, everything is built around GameObjects and Components, which are controlled via scripts.
	
	\begin{lstlisting}
		// A simple script to move a player in Unity
		using UnityEngine;
		
		public class PlayerController : MonoBehaviour
		{
			public float speed = 5f;
			
			void Update()
			{
				float moveHorizontal = Input.GetAxis("Horizontal");
				float moveVertical = Input.GetAxis("Vertical");
				
				Vector3 movement = new Vector3(moveHorizontal, 0.0f, moveVertical);
				transform.Translate(movement * speed * Time.deltaTime);
			}
		}
	\end{lstlisting}
	
	\chapter{C\# Basics for Game Development}
	\section{C\# Syntax Overview}
	\subsection{Variables and Data Types}
	C\# supports various data types such as integers, floats, booleans, and strings. These are used to store game states such as health, position, and player actions.
	
	\textbf{Example:}
	\begin{lstlisting}
		int health = 100;
		float speed = 10.5f;
		bool isJumping = false;
		string playerName = "Hero";
	\end{lstlisting}
	
	\subsection{Operators and Control Flow}
	Control the flow of your game using operators, if-else conditions, and loops.
	
	\begin{lstlisting}
		// If-else for game logic
		if (health <= 0)
		{
			GameOver();
		}
		else
		{
			ContinuePlaying();
		}
	\end{lstlisting}
	
	\subsection{Loops (For, While, Do-While)}
	\begin{lstlisting}
		// Using a loop to spawn enemies
		for (int i = 0; i < 5; i++)
		{
			SpawnEnemy();
		}
	\end{lstlisting}
	
	\section{Collections and Data Structures}
	C\# provides several built-in data structures such as arrays, lists, and dictionaries to manage game elements such as players, enemies, and items.
	\subsection{Arrays and Lists}
	Arrays and lists are useful for storing multiple items of the same type.
	
	\begin{lstlisting}
		// Arrays to store multiple player scores
		int[] playerScores = new int[10];
		
		// List to store enemies in the game
		List<Enemy> enemies = new List<Enemy>();
	\end{lstlisting}
	
	\subsection{Dictionaries}
	Dictionaries allow you to store key-value pairs, which can be useful for mapping items or players to certain properties.
	
	\begin{lstlisting}
		// Dictionary mapping player names to their score
		Dictionary<string, int> playerScores = new Dictionary<string, int>();
		playerScores.Add("Player1", 100);
	\end{lstlisting}
	
	\chapter{Object-Oriented Programming in C\#}
	\section{Classes and Objects in Game Development}
	In game development, classes are used to represent game entities such as players, enemies, and weapons. Objects are instances of these classes.
	
	\begin{lstlisting}
		public class Player
		{
			public string Name { get; set; }
			public int Health { get; set; }
			public float Speed { get; set; }
			
			public Player(string name, int health, float speed)
			{
				Name = name;
				Health = health;
				Speed = speed;
			}
			
			public void Move()
			{
				Console.WriteLine($"{Name} is moving at speed {Speed}");
			}
		}
	\end{lstlisting}
	
	\section{Inheritance and Polymorphism in Games}
	Inheritance allows you to create a class hierarchy, where more specialized classes (like Enemy or Boss) inherit properties and methods from a base class (like Character).
	
	\begin{lstlisting}
		public class Enemy : Player
		{
			public int Damage { get; set; }
			
			public Enemy(string name, int health, float speed, int damage)
			: base(name, health, speed)
			{
				Damage = damage;
			}
			
			public void Attack()
			{
				Console.WriteLine($"{Name} attacks and deals {Damage} damage.");
			}
		}
	\end{lstlisting}
	
	\chapter{Advanced C\# Concepts for Game Development}
	\section{Delegates and Events}
	Delegates and events are powerful features in C\# for managing interactions in game components. Events are used for communication between objects, such as triggering actions when a player completes a level or when health reaches zero.
	
	\begin{lstlisting}
		// Declare a delegate for handling game events
		public delegate void GameEvent();
		
		public class GameManager
		{
			public static event GameEvent OnGameOver;
			
			public void GameOver()
			{
				if (OnGameOver != null)
				{
					OnGameOver();
				}
			}
		}
		
		// Subscribe to the event
		public class Player
		{
			public Player()
			{
				GameManager.OnGameOver += EndGame;
			}
			
			void EndGame()
			{
				Console.WriteLine("Game Over!");
			}
		}
	\end{lstlisting}
	
	\chapter{Working with Unity}
	\section{GameObject and Components}
	GameObjects are the most important building blocks in Unity. They represent all objects in your scene, including characters, items, and scenery.
	
	\begin{lstlisting}
		public class Rotator : MonoBehaviour
		{
			void Update()
			{
				transform.Rotate(new Vector3(15, 30, 45) * Time.deltaTime);
			}
		}
	\end{lstlisting}
	
	\section{Physics and Collisions}
	Unity's physics engine allows you to create realistic movements and interactions between game objects. To utilize physics, you need to understand Rigidbody and Collider components.
	
	\subsection{Rigidbody Component}
	The Rigidbody component adds physical properties to a GameObject, enabling it to be affected by forces, gravity, and collisions.
	
	\begin{lstlisting}
		// Adding Rigidbody to a GameObject
		void Start()
		{
			Rigidbody rb = gameObject.AddComponent<Rigidbody>();
			rb.mass = 5.0f;
		}
	\end{lstlisting}
	
	\subsection{Collider Component}
	Colliders define the shape of a GameObject for physical collisions. Different types of colliders (Box, Sphere, Capsule) can be used based on your requirements.
	
	\begin{lstlisting}
		// Using BoxCollider for a cube GameObject
		void Start()
		{
			BoxCollider boxCollider = gameObject.AddComponent<BoxCollider>();
			boxCollider.size = new Vector3(1, 1, 1);
		}
	\end{lstlisting}
	
	\subsection{Collision Detection}
	Unity allows you to detect collisions using methods like `OnCollisionEnter`, `OnCollisionStay`, and `OnCollisionExit`.
	
	\begin{lstlisting}
		void OnCollisionEnter(Collision collision)
		{
			if (collision.gameObject.CompareTag("Player"))
			{
				Debug.Log("Collision with Player detected!");
			}
		}
	\end{lstlisting}
	
	\chapter{User Input in Games}
	User input is essential for making games interactive. In Unity, you can capture user input from keyboards, mice, and controllers.
	
	\section{Capturing Keyboard Input}
	Unity provides easy access to keyboard input through the `Input` class. You can check for specific keys or axes.
	
	\begin{lstlisting}
		// Checking for keyboard input in Update
		void Update()
		{
			if (Input.GetKeyDown(KeyCode.Space))
			{
				Jump();
			}
		}
	\end{lstlisting}
	
	\section{Mouse Input}
	You can also capture mouse clicks and movements in Unity.
	
	\begin{lstlisting}
		// Detecting mouse button click
		void Update()
		{
			if (Input.GetMouseButtonDown(0))
			{
				Debug.Log("Left mouse button clicked!");
			}
		}
	\end{lstlisting}
	
	\section{Controller Input}
	To handle controller input, you can use the `Input` class with defined axes.
	
	\begin{lstlisting}
		// Getting controller input
		float horizontal = Input.GetAxis("Horizontal");
		float vertical = Input.GetAxis("Vertical");
		Vector3 movement = new Vector3(horizontal, 0, vertical);
		transform.Translate(movement * speed * Time.deltaTime);
	\end{lstlisting}
	
	\chapter{Game Design Principles}
	Understanding game design principles is crucial for developing engaging and fun games.
	
	\section{Game Mechanics}
	Game mechanics refer to the rules and systems that govern gameplay. This includes how players interact with the game and the objectives they must achieve.
	
	\subsection{Core Gameplay Loop}
	The core gameplay loop is a sequence of actions that players perform repetitively. It often consists of actions such as exploring, collecting resources, and battling enemies.
	
	\section{Level Design}
	Level design involves creating the environments where players interact. A well-designed level provides challenges and opportunities for players to engage with game mechanics.
	
	\section{User Interface (UI) Design}
	A good UI enhances player experience by providing clear feedback and options. Unity provides tools for designing UI elements.
	
	\begin{lstlisting}
		// Example of creating a simple UI button in Unity
		public Button myButton;
		
		void Start()
		{
			myButton.onClick.AddListener(OnButtonClick);
		}
		
		void OnButtonClick()
		{
			Debug.Log("Button clicked!");
		}
	\end{lstlisting}
	
	\chapter{Debugging and Testing}
	Debugging is an essential skill for game developers. Unity provides several tools to help you find and fix issues in your game.
	
	\section{Using Debug.Log}
	Use `Debug.Log` to print messages to the console, which can help track down issues.
	
	\begin{lstlisting}
		void Update()
		{
			Debug.Log("Current health: " + health);
		}
	\end{lstlisting}
	
	\section{Unit Testing in C\#}
	Unit testing allows you to test individual components of your game to ensure they work as expected. Use a framework like NUnit for writing tests.
	
	\begin{lstlisting}
		// Example of a simple unit test
		[Test]
		public void TestPlayerHealth()
		{
			Player player = new Player("Hero", 100, 5);
			Assert.AreEqual(100, player.Health);
		}
	\end{lstlisting}
	
	\chapter{Optimizing Game Performance}
	Performance optimization is crucial in game development to ensure a smooth experience for players. 
	
	\section{Understanding Performance Bottlenecks}
	Analyze your game's performance using Unity's Profiler to identify bottlenecks in rendering, memory usage, and CPU/GPU usage.
	
	\section{Memory Management}
	Manage memory efficiently by avoiding excessive object creation during runtime. Use object pooling to recycle objects instead of creating new instances.
	
	\begin{lstlisting}
		// Simple object pooling implementation
		public class ObjectPool
		{
			private List<GameObject> pool;
			
			public ObjectPool(GameObject prefab, int size)
			{
				pool = new List<GameObject>();
				for (int i = 0; i < size; i++)
				{
					GameObject obj = GameObject.Instantiate(prefab);
					obj.SetActive(false);
					pool.Add(obj);
				}
			}
			
			public GameObject GetObject()
			{
				foreach (var obj in pool)
				{
					if (!obj.activeInHierarchy)
					{
						obj.SetActive(true);
						return obj;
					}
				}
				return null; // Or expand pool
			}
		}
	\end{lstlisting}
	
	\chapter{Conclusion}
	C\# is a powerful language for game development, especially when paired with Unity. This guide has introduced you to the fundamentals of C\# programming, game design principles, user input handling, debugging techniques, and optimization strategies. 
	
	\section{Next Steps}
	To further enhance your skills, consider:
	\begin{itemize}
		\item Creating small projects to apply what you've learned.
		\item Participating in game jams to practice rapid development.
		\item Exploring advanced topics such as AI programming and shader development.
	\end{itemize}
	
\end{document}
	

